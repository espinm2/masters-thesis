%%%%%%%%%%%%%%%%%%%%%%%%%%%%%%%%%%%%%%%%%%%%%%%%%%%%%%%%%%%%%%%%%%%
%                                                                 %
%                            ABSTRACT                             %
%                                                                 %
%%%%%%%%%%%%%%%%%%%%%%%%%%%%%%%%%%%%%%%%%%%%%%%%%%%%%%%%%%%%%%%%%%%

\specialhead{ABSTRACT}

% --------------------------------------------
% Introduction to the problem
% --------------------------------------------
Daylighting plays an important role in architecture, its creative and effective use offers aesthetics visuals, natural light, reduced energy consumption. 
Additional studies reviled many health benefits of having exposure to natural lighting over fluorescent lighting typical in office spaces. %TODO Cite
Due to these advantages of natural lighting, architects incorporate daylighting into the design of spaces such as offices, museums, and even residential estates. %TODO Cite
Daylighting is considered early on in the design process, where the use of sophisticated 3D modeling softwares can be considered excessive for the simple geometries considered in the early design phase.
Previous research has been conducted on an augmented reality physical sketching interface that allows users through placement of physical primitives to generate 3D models interactively for daylighting simulation.
User studies have shown this system, the Virtual Heliodon, to be useful in the creation of 3D spaces and analysis of daylighting performance.%TODO Cite
Despite the advantages offered, the physical setup of the Virtual Heliodon requires the use of many projectors, physical props, and a controlled lighting enviroment.

% --------------------------------------------
% Solution to the problem
% --------------------------------------------
While the Virtual Heliodon's augmented reality display offers users a collaborative space to share and experiment on architectural designs, the overhead of the setup of the Virtual Heliodon make it inaccessible to a potentially large user base that could provide us invaluable feedback on key algorithms and features that would improve daylighting analysis and design. 
We have come up with a web based interface that will be accessible to the general public and preserves the spirit the Virtual Heliodon.
This interface will offer most features available in the Virtual Heliodon in addition to features exclusive to the web application.%wording
Most importantly, we can reach a larger audience with our web based application to improve on the daylighting and sketch interpretation algorithms that are key to Virtual Heliodon.
% --------------------------------------------
% Do we have any final words?
% --------------------------------------------
In this thesis we describe in detail the features translated and exlusive to the web application of the Virtual Heliodon.
We will also discuss how we conducted a prototype study on the usabilty of our web application and the results from that study.

% --------------------------------------------
% Didn't make the CUT
% --------------------------------------------
\iffalse
Short list of what offer right now over physical heliodon
In addition using WebGL to render the visualization client side we can offer the user a view from any location in the scene without obscuring portions of the model.
Most importantly, we can reach a larger audience with our web based application to improve on the daylighting and sketch interpretation algorithms that are key to the Virtual Heliodon.
Despite the advantages offered, the physical setup of the Virtual Heliodon requires 4 or more HD projectors, a strong metal chaste, a high resolution camera,a large set of physical props, and a controlled lighting environment. 
\fi




